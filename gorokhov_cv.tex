\documentclass{resume}  % Use the custom resume.cls style

\usepackage{cmap}  % поиск в PDF
\usepackage{mathtext}  % русские буквы в формулах
\usepackage[T2A]{fontenc}  % кодировка
\usepackage[utf8]{inputenc}  % кодировка исходного текста
\usepackage[english,russian]{babel}  % локализация и переносы
\usepackage{indentfirst}
\usepackage[left=0.4 in,top=0.4in,right=0.4 in,bottom=0.4in]{geometry}  % Document margins
\newcommand{\tab}[1]{\hspace{.2667\textwidth}\rlap{#1}} 
\newcommand{\itab}[1]{\hspace{0em}\rlap{#1}}

%----------------------------------------------------------------------------------------
% CONTACTS
%----------------------------------------------------------------------------------------

\name{Владислав Горохов}
\address{+7(985) 034-0348 \\ Москва, РФ} 
\address{\href{mailto:gorokhovvd@icloud.com}{gorokhovvd@icloud.com} \\ \href{https://www.linkedin.com/in/vladislav-gorokhov-837a69217}{LinkedIn} \\
\href{https://github.com/VladOnMyOwn}{GitHub}}

\begin{document}

%----------------------------------------------------------------------------------------
% SKILLS
%----------------------------------------------------------------------------------------

\begin{rSection}{Навыки}
   \begin{tabular}{ @{} >{\bfseries}l @{\hspace{6ex}} l }
      Ключевые    & Машинное обучение, MLOps, Глубинный анализ данных, Временные ряды, Эконометрика,             \\
                  & Многомерная статистика, Математическая статистика, Prompt engineering, CI/CD, Scrum
      \\
      Технические & Python (LightGBM, XGBoost, optuna, scikit-learn, statsmodels, feature-engine, numpy, scipy,  \\
                  & pandas, asyncio, FastAPI, pydantic, PyTest, loguru, tox, lxml и др.), R, Bash, Git, DVC, S3, \\
                  & MLflow, PostgreSQL, MySQL, Docker
      \\
      Языки       & Английский - C1
      \\
   \end{tabular}\\
\end{rSection}

%----------------------------------------------------------------------------------------
% WORK EXPERIENCE
%----------------------------------------------------------------------------------------

\begin{rSection}{Опыт}
   \textbf{Data Scientist / ML Engineer} \hfill Январь 2023 - Настоящее время\\
   Zaymigo \hfill \textit{Москва, РФ}
   \begin{itemize}
      \itemsep -5pt {}
      \item Увеличил значение метрики ROI на 2 п.п., снизил долю просрочек FPD30 на 4 п.п. для портфеля первичных клиентов, а также увеличил ROI на 2 п.п., снизил долю FPD30 на 2 п.п. для портфеля повторных клиентов путем разработки и вывода в продуктовую среду новых ML-моделей для системы кредитного скоринга
      \item Сократил время обнаружения дрейфа данных в 4 раза, разработав систему мониторинга и тестирования качества данных и ML-моделей на основе Evidently AI
   \end{itemize}
\end{rSection}

%----------------------------------------------------------------------------------------
%  EDUCATION
%----------------------------------------------------------------------------------------

\begin{rSection}{Образование}
   {\bf Прикладная математика и информатика - Бакалавр}, РЭУ им. Г.В. Плеханова \hfill {2019 - 2023}\\
   Диплом с отличием
\end{rSection}

\begin{rSection}{Курсы}
   \begin{itemize}
      \itemsep -6pt {}
      \item \href{https://www.udemy.com/certificate/UC-5696cb94-0706-4577-9666-2f3f6b4ace5f/}{Deployment of Machine Learning Models}. Udemy, 2024
      \item \href{https://jovian.com/certificate/MFQTOOJUGA}{Data Structures and Algorithms in Python}. Jovian, 2022
      \item \href{https://www.hackerrank.com/certificates/cfb6b0d6f9fa}{SQL Advanced}. HackerRank, 2022
      \item \href{https://www.datacamp.com/completed/statement-of-accomplishment/course/545c2bf2475d2e684ed8869b624a8af044d56621}{Customer Analytics and A/B Testing in Python}. DataCamp, 2022
      \item \href{https://www.coursera.org/account/accomplishments/certificate/KXUHDPADQSW7}{Agile Project Management}. Google, Coursera, 2021
   \end{itemize}
\end{rSection}

%----------------------------------------------------------------------------------------
%  HACKATHONS
%----------------------------------------------------------------------------------------

\begin{rSection}{Хакатоны}
   \begin{itemize}
      \itemsep -5pt {}
      \item Хакатон СФО Цифровой прорыв. Сезон: Искусственный интеллект, 2023. 3 место в кейсе \href{https://github.com/orgs/OBeMeHacks/repositories}{"Проанализируй финансовое поведение"}\ от НПФ Будущее. Выступал в роли капитана и наставника команды, DS/MLE
      \item Международный хакатон Цифровой прорыв. Сезон: Искусственный интеллект, 2023. 11 место в кейсе \href{https://github.com/orgs/obm-hacks/repositories}{"Оценка эффективности вложений в содержание недвижимости"}\ от Банка России. Выступал в роли капитана команды, DS/MLE
      \item Всероссийский хакатон Цифровой прорыв. Сезон: Искусственный интеллект, 2023. 11 место в кейсе \href{https://github.com/VladOnMyOwn/hacks_ai_dsc}{"Построй свою универсальную рекомендательную систему"}\ от Digital Consulting Solutions. Выступал в роли MLE
   \end{itemize}
\end{rSection}

%----------------------------------------------------------------------------------------
%  ARTICLES AND CONFERENCES
%----------------------------------------------------------------------------------------

\begin{rSection}{Конференции}
   \begin{itemize}
      \itemsep -5pt {}
      \item \href{https://elibrary.ru/ipezcy}{Пуассоновский бутстрэп как метод ускорения статистических тестов на больших данных}, Москва, 2022. Диплом 1-й степени
      \item \href{https://os-russia.com/SBORNIKI/KON-449-NC.pdf}{Прогнозирование курса Биткоина с применением методов и технологий эконометрического и нейросетевого моделирования}, Таганрог, 2022
   \end{itemize}
\end{rSection}

\end{document}
